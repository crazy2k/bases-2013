\section{Resultados}

\textsl{En esta secci\'on presentaremos los resultados obtenidos 
al utilizar diferentes trazas.}

\vspace*{0.3cm}

Se presenta para cada traza el hit rate de cada buffer.

\subsection{Trazas utilizadas}

Las trazas que utilizamos fueron las generadas utilizando la
clase \texttt{MainTraceGenerator.java} provista por la cátedra.

% \begin{itemize}
%     \item
%             \textbf{Random Typical}
%             Se ejecuta dos veces una transacci\'on sobre 100 bloques
%             distinto en una tabla de 200 bloques.
            
%     \item
%             \textbf{Random}
%             Se realizan 1000 accesos random a una tabla de 200 bloques.
            
%     \item
%             \textbf{Read - File Scan - Read (RFR)}
%             Se realiza dos veces una lectura sobre los mismos 100 
%             bloques, teniendo un file scan de 200 bloques entremedio
%             de la dos lecturas.
            
%     \item
%             \textbf{Read - Index Scan - Read (RIR)}
%             Igual al caso anterior pero con un index scan en vez de 
%             un file scan.
% \end{itemize}


Dichas trazas pueden encontrarse en la carpeta 
\texttt{ubadb/generated}, y los archivos son
% \texttt{random\_typical.trace}, 
% \texttt{random-1000.trace}, 
% \texttt{rfr.trace} y
% \texttt{rir.trace}.