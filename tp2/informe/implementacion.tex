\section{Detalles de Implementaci\'on}

\textsl{En esta secci\'on se explican algunos detalles de las
implementaciones de los algoritmos.}

\vspace*{0.5cm}
%~ 
%~ Se dese\'o implementar un Buffer Pool que delegara el almacenamiento 
%~ de cada página en distintos pools. 
%~ Esto se logró con \texttt{MultipleBufferPool}, que mantiene una lista 
%~ de \texttt{BufferPoolAssignment}, cada uno de los cuales 
%~ \textit{acepta} algunas páginas y otras no.
%~ Para aquellas que acepta, les asigna un buffer pool.
%~ 
%~ \vspace*{0.3cm}
%~ 
%~ En nuestro caso particular utilizamos un pool por tabla, por lo
%~ cual tenemos \\
%~ \texttt{TableToBufferPoolAssignment} que acepta todas 
%~ las páginas de una tabla en particular.
%~ Tenemos tambi\'en una subclase de \texttt{MultipleBufferPool} 
%~ (\texttt{PerTableMultipleBufferPool}) que recibe una lista de 
%~ definiciones de pools (\texttt{TablePool}); es decir, las 
%~ tablas para las que creará un pool, el porcentaje del tamaño
%~ total que asignará a cada tabla y la estrategia de reemplazo
%~ de cada pool.


Para comenzar con la implementación se comenzó por pasar por todo el proyecto 
provisto por la cátedra, para decidir qué iba a ser necesario implementar.

Luego de esta revisión, se llegó a la conclusión de que se iba a necesitar 
extender el tableDescriptor, agregándole un campo Buffer Pool.

