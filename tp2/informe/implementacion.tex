\section{Detalles de Implementaci\'on}

\textsl{En esta secci\'on se explican algunos detalles de las
implementaciones de los algoritmos.}

% Se muestran las secciones de código que se consideran relevantes.

\vspace*{0.5cm}

% Para comenzar con la implementación se comenzó por pasar por todo el proyecto 
% provisto por la cátedra, para decidir qué iba a ser necesario implementar.
% 
% Luego de esta revisión, se llegó a la conclusión de que se iba a necesitar 
% extender el tableDescriptor, agregándole un campo Buffer Pool.
% 


\subsection{PoolDescriptor}

Teniendo en cuenta la idea de la clase \texttt{TableDescriptor},
creamos la clase \texttt{PoolDescriptor}, que almacena el 
nombre y el tamaño de un pool.

\begin{verbatimtab}[4]
public class PoolDescriptor
{
	private String name;
	private Integer size;

	...
}
\end{verbatimtab}



\subsection{Catalog}

Agregamos una lista de \texttt{PoolDescriptor} a la ya existente lista 
de \texttt{TableDescriptor}.
Agregamos un método que, dado el nombre de un Pool devuelve su tamaño, 
buscándolo en la lista poolDesciptors.

\begin{verbatimtab}[4]
public class Catalog
{
	private List<PoolDescriptor> poolDescriptors;
	private List<TableDescriptor> tableDescriptors;	

	....

	public Integer getSizeOfPool(String name) throws Exception 
	{
		for (PoolDescriptor p:poolDescriptors)
		{
			if (p.getName() == name)
			{
				return p.getSize();
			}
		}
		
		throw new Exception("Name does not exists");
	}
}
\end{verbatimtab}

