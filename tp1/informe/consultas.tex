\section{Consultas}

A continuación se presentan las consultas implementadas en SQL.

\subsection*{Consulta 1}

Mediante SQL escribir una consulta para obtener el nombre, apellido y número identificatorio
de aquellos pasajeros que han viajado a todos los países cubiertos por la línea aérea en los
últimos 5 años.


\begin{small}
\begin{verbatimtab}[2]

SELECT dp.nombre, dp.apellido, u.id_user
FROM Usuario u, DatosPersonales dp
WHERE NOT EXISTS
(
    SELECT p.id_pais, p.nombre
    FROM Vuelo v1, Aeropuerto a1, Ciudad c1, Pais p
    WHERE (v1.a_salida = a1.id_a or v1.a_llegada = a1.id_a)
    AND a1.id_ciudad = c1.id_ciudad
    AND c1.id_pais = p.id_pais
    AND NOT EXISTS
    (
	      SELECT *
	      FROM Hace h, Reserva r, Tiene t, Servicio s, Vuelo v, 
		      Aeropuerto a, Ciudad c, Estado e
	      WHERE u.id_user = h.id_user 
	      AND h.id_reserva = r.id_reserva 
	      AND r.id_reserva = t.id_res 
	      AND t.id_serv = s.id_servicio
	      AND s.id_vuelo = v.id_vuelo 
	      AND (v.a_llegada = a.id_a or v.a_salida = a.id_a) 
	      AND a.id_ciudad = c.id_ciudad 
	      AND c.id_pais = p.id_pais
	      AND r.id_estado = 3
	      AND ((s.salida >= date_sub(now(), interval 5 year)) 
		      OR (s.llegada >= date_sub(now(), interval 5 year)))  
    )
)
AND u.id_user = dp.id_user

\end{verbatimtab}
\end{small}

\newpage

\subsection*{Consulta 2}

Obtener un reporte que como mínimo contenga: Todos los códigos identificatorios de los
aeropuertos, un periodo de tiempo de la forma año/mes, la cantidad de pasajeros que
ascendieron y descendieron en ese aeropuerto durante el periodo. El reporte debe estar
ordenado por la cantidad total de personas que viajaron. Y debe ser ejecutado para un rango
de fechas. Aclaración: dependiendo del rango de fechas, para un mismo aeropuerto pueden
aparecer varios periodos distintos.

\begin{small}
\begin{verbatimtab}[2]

\end{verbatimtab}
\end{small}


\newpage

\subsection*{Consulta 3}

Controlar mediante alguna restricción que un usuario no pueda realizar reservas que se
superpongan en el tiempo. La única excepción que se permite consiste en que un usuario
puede realizar a lo sumo 2 reservas para la misma fecha de viaje entre los mismos
aeropuertos de origen y destino, siempre que la fecha de partida no se encuentre dentro de
los próximos 7 días.

\begin{small}
\begin{verbatimtab}[2]
CREATE TRIGGER chequear_reservas_superpuestas
BEFORE INSERT ON Tiene
FOR EACH ROW
begin
	IF EXISTS (
		SELECT h1.id_reserva, h1.id_user
		FROM Hace h1, Servicio s1, Usuario u1, Reserva r1
		WHERE h1.id_user = u1.id_user AND
			h1.id_reserva = r1.id_reserva AND
			NEW.id_res = r1.id_reserva AND
			NEW.id_serv = s1.id_servicio
			AND EXISTS(
			    SELECT s2.salida
			    FROM Servicio s2, Tiene t2, Reserva r2, Hace h2
			    WHERE h2.id_user = u1.id_user
				AND h2.id_reserva <> h1.id_reserva
				AND h2.id_reserva = r2.id_reserva
				AND t2.id_res = r2.id_reserva
				AND t2.id_serv = s2.id_servicio
				AND s2.salida = s1.salida
				AND s1.id_servicio <> s2.id_servicio
				AND s2.salida NOT IN (
					    (SELECT s3.salida
					    FROM Reserva r3, Hace h3, Tiene t3, Servicio s3, Usuario u3, Vuelo v3
					    WHERE r3.id_reserva = h3.id_reserva
						    AND r3.id_reserva = t3.id_res
						    AND u3.id_user = h3.id_user
						    AND t3.id_serv = s3.id_servicio
						    AND s3.id_vuelo = v3.id_vuelo
						    AND s3.salida IN (
								    SELECT s4.salida
								    FROM Servicio s4, Hace h4, Tiene t4, Reserva r4, Vuelo v4
								    WHERE r4.id_reserva <> r3.id_reserva
									    AND h4.id_reserva = r4.id_reserva
									    AND h4.id_user = h3.id_user
									    AND r4.id_reserva = t4.id_res
									    AND t4.id_serv = s4.id_servicio
									    AND s4.id_vuelo = v4.id_vuelo
									    AND v4.a_salida = v3.a_salida
									    AND v4.a_llegada = v4.a_llegada
									    AND datediff(s4.salida, now()) > 7)
								    ) 
							    )
	)) THEN
		SIGNAL SQLSTATE '45000'
			SET MESSAGE_TEXT = 'No se puede insertar la reserva';
	END IF;
end;
\end{verbatimtab}
\end{small}


Para poder imponer esta restricción, usamos un trigger, que cuando ocurre el evento: 
insertar en la tabla Tiene (tabla que relaciona las reservas con los servicios), se fija
que la reserva a la cual está asociada esta tabla no tenga como fecha de salida de vuelo
la misma fecha de salida de otras reservas del mismo usuario, salvo que la fecha de partida
del vuelo no sea dentro de la próxima semana. En caso de que tengta la misma fecha de salida
se dispara una excepción inexistente de manera tal que no se concreta la operación.

\newpage

\subsection*{Consulta 4}

Para las reservas superpuestas que se permiten en el punto anterior, se debe contar con una
funcionalidad que cancele una de las reservas duplicadas (la más económica) cuando la
fecha de partida se encuentre dentro de los próximos 7 días.


\begin{small}
\begin{verbatimtab}[2]

\end{verbatimtab}
\end{small}
